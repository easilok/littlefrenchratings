\documentclass[11pt, a4paper, sans]{article}
%% For iso-8859-1 (latin1), comment next line and uncomment the second line
\usepackage[utf8]{inputenc}
%\usepackage[latin1]{inputenc}

\title{Especificação Plataforma de Avaliação de Francesinhas}
\date{2020-02-02}
\author{Luís Pereira}

\begin{document}
	\pagenumbering{gobble}
	\maketitle
	\newpage
	\pagenumbering{arabic}

	\section{Introdução}
	
	O objetivo desta plataforma será gerir marcação de provas de pratos gastronómicos e posterior
	avaliação dos mesmos. A plataforma terá registos de estabelecimentos e seus pratos. Será possível marcar
	uma prova de um prato, associando utilizadores ao prato em determinada data. 

	Cada utilizador terá a sua página pessoal onde mostrará as provas seguintes, bem como um histórico das 
	suas avaliações.

	Será possível avaliar cada prato em vários critérios, que serão configuráveis e expansíveis.

	Na página pessoal de cada utilizador, as provas terão um \textit{url} na morada/gps que permitirá 
	facilmente abrir o gps para a deslocação ao estabelecimento correspondente.

	A plataforma será criada para uma rota de provas de francesinhas, mas poderá facilmente ser 
	usado em qualquer prato.

	\section{Modos de Interação}

	\subsection{Utilizador Base}

	Todos os utilizadores registados poderão fazer as operações:

	\begin{itemize}
		\item \underline{\textbf{Gerir Marcações:}} Criar uma marcação de prova para si, ou cancelar uma existente;
		\item \underline{\textbf{Gerir Avaliações:}} Criar e editar uma avaliação feita a um prato.
	\end{itemize}

	\subsection{Utilizador Avançado}

	Os utilizadores avançados recebem algumas novas permissões face ao base:

	\begin{itemize}
		\item \underline{\textbf{Gerir Estabelecimentos:}} Criar e editar estabelecimentos na plataforma.
		\item \underline{\textbf{Gerir Pratos:}} Criar e editar pratos e associar fotos ao mesmo.
		\item \underline{\textbf{Gerir Marcações:}} Criar uma marcação com qualquer utilizador da plataforma;
	\end{itemize}

	\subsection{Administração}
	Os administradores da plataforma terão as permissões totais, adicionando:
	\begin{itemize}
		\item \underline{\textbf{Gerir Critérios Avaliação:}} Criar e editar os critérios de avaliação de pratos.
		\item \underline{\textbf{Gerir Utilizadores:}} Os utilizadores serão de registo por pedido. Os administradores
			terão a capacidade de adicionar e editar todos os utilizadores e suas permissões.
		\item \underline{\textbf{Moderar Avaliações:}} Apagar avaliações que sejam desajustadas.
	\end{itemize}


\end{document}

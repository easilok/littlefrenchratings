\documentclass[11pt, a4paper, sans]{article}
%% For iso-8859-1 (latin1), comment next line and uncomment the second line
\usepackage[utf8]{inputenc}
%\usepackage[latin1]{inputenc}

\title{Especificação Plataforma de Avaliação de Francesinhas}
\date{2020-02-02}
\author{Luís Pereira}

\begin{document}
	\pagenumbering{gobble}
	\maketitle
	\newpage
	\pagenumbering{arabic}

	\section{Introdução}
	
	O objetivo desta plataforma será gerir marcação de provas de pratos gastronómicos e posterior
	avaliação dos mesmos. A plataforma terá registos de estabelecimentos e seus pratos. Será possível marcar
	uma prova de um prato, associando utilizadores ao prato em determinada data. 

	Cada utilizador terá a sua página pessoal onde mostrará as provas seguintes, bem como um histórico das 
	suas avaliações.

	Será possível avaliar cada prato em vários critérios, que serão configuráveis e expansíveis.

	Na página pessoal de cada utilizador, as provas terão um \textit{url} na morada/gps que permitirá 
	facilmente abrir o gps para a deslocação ao estabelecimento correspondente.

	A plataforma será criada para uma rota de provas de francesinhas, mas poderá facilmente ser 
	usado em qualquer prato.

	\section{Modos de Interação}

	\subsection{Utilizador Base}

	Todos os utilizadores registados poderão fazer as operações:

	\begin{itemize}
		\item \underline{\textbf{Gerir Marcações:}} Criar uma marcação de prova para si, ou cancelar uma existente;
		\item \underline{\textbf{Gerir Avaliações:}} Criar e editar uma avaliação feita a um prato.
	\end{itemize}

	\subsection{Utilizador Avançado}

	Os utilizadores avançados recebem algumas novas permissões face ao base:

	\begin{itemize}
		\item \underline{\textbf{Gerir Estabelecimentos:}} Criar e editar estabelecimentos na plataforma.
		\item \underline{\textbf{Gerir Pratos:}} Criar e editar pratos e associar fotos ao mesmo.
		\item \underline{\textbf{Gerir Marcações:}} Criar uma marcação com qualquer utilizador da plataforma;
	\end{itemize}

	\subsection{Administração}
	Os administradores da plataforma terão as permissões totais, adicionando:
	\begin{itemize}
		\item \underline{\textbf{Gerir Critérios Avaliação:}} Criar e editar os critérios de avaliação de pratos.
		\item \underline{\textbf{Gerir Utilizadores:}} Os utilizadores serão de registo por pedido. Os administradores
			terão a capacidade de adicionar e editar todos os utilizadores e suas permissões.
		\item \underline{\textbf{Moderar Avaliações:}} Apagar avaliações que sejam desajustadas.
	\end{itemize}

	\section{Base de Dados}

	Nesta secção está descrita a estrutura da base de dados da plataforma.

	\subsection{Tabela: \textit{Users}} \label{db-table-users}

	Registo de todos os utilizadores da plataforma:

	\begin{itemize}
		\item \textbf{\textit{name}} - Nome do utilizador
		\item \textbf{\textit{email}} - Email de registo. Serve também como \underline{\textit{username}} para 
			iniciar sessão na plataforma
		\item \textbf{\textit{password}} - Nome do utilizador
	\end{itemize}

	\subsection{Tabela: \textit{Roles}} \label{db-table-roles}

	Tabela auxiliar para grupos de permissão de utilizador:

	\begin{itemize}
		\item \textbf{\textit{name}} - Nome da permissão para visualização ao atribuir a utilizador.
		\item \textbf{\textit{role}} - Valor numérico usado no servidor para definir qual a permissão dada.
		\item \textbf{\textit{enable}} - Permite desabilitar a permissão no processamento da plataforma. 
			De momento não está a ser usado.
	\end{itemize}

	\subsection{Tabela: \textit{Role\_User}} \label{db-table-role-user}

	Atribui o grupo de permissão de cada utilizador registado:

	\begin{itemize}
		\item \textbf{\textit{user\_id}} - Identificador do utilizador na tabela \underline{\textit{users}}\textsuperscript{\ref{db-table-users}}. Um utilizador pode ter apenas uma permissão (N:1).
		\item \textbf{\textit{role\_id}} - Identificador da permissão na tabela \underline{\textit{roles}}\textsuperscript{\ref{db-table-roles}}. Uma permissão pode ter vários utilizadores (1:N).
	\end{itemize}

	\subsection{Tabela: \textit{Configurations}} \label{db-table-configurations}

	Configurações da plataforma passíveis de ser configuradas por utilizador.

	\begin{itemize}
		\item \textbf{\textit{name}} - Nome da configuração a ser apresantado na página de configurações.
		\item \textbf{\textit{default}} - Valor inicial da configuração.
		\item \textbf{\textit{role\_id}} - Identificador da permissão na tabela \underline{\textit{roles}}\textsuperscript{\ref{db-table-roles}}. Uma configuração está associada a um dos grupos de permissão (N:1).
		\item \textbf{\textit{enable}} - Permite desabilitar este tipo de configuração.
	\end{itemize}

	\subsection{Tabela: \textit{Establishments}} \label{db-table-establishments}

	Caraterísticas de cada estabelecimento criado na plataforma.

	\begin{itemize}
		\item \textbf{\textit{name}} - Nome do estabelecimento.
		\item \textbf{\textit{address}} - Morada do estabelecimento.
		\item \textbf{\textit{parish}} - Freguesia do estabelecimento.
		\item \textbf{\textit{city}} - Cidade do estabelecimento.
		\item \textbf{\textit{gps}} - Coordenadas GPS da localização do estabelecimento.
		\item \textbf{\textit{telephone}} - Contato Telefónico principal do estabelecimento.
		\item \textbf{\textit{telephone2}} - Um contato telefónico alternativo do estabelecimento.
		\item \textbf{\textit{telephone3}} - Um outro contato telefónico alternativo do estabelecimento.
		\item \textbf{\textit{email}} - Contato email do estabelecimento.
		\item \textbf{\textit{card}} - Valor boleano que identifica se o estabelecimento permite pagamento
			multibanco..
		\item \textbf{\textit{timetable}} - Horário de funcionamento do estabelecimento.
		\item \textbf{\textit{obs}} - Outras observações que o criador queira identificar no estabelecimento.
		\item \textbf{\textit{user\_id}} - Identificador do utilizador na tabela 
			\underline{\textit{users}}\textsuperscript{\ref{db-table-users}} 
			Um estabelecimento pode ter apenas um utilizador que o criou (N:1).
	\end{itemize}

	\subsection{Tabela: \textit{Plates}} \label{db-table-plates}

	Pratos de um estabelecimento criados na plataforma.

	\begin{itemize}
		\item \textbf{\textit{name}} - Nome do prato.
		\item \textbf{\textit{price}} - Preço médio do prato.
		\item \textbf{\textit{obs}} - Outras observações que o criador queira identificar no prato.
		\item \textbf{\textit{user\_id}} - Identificador do utilizador na tabela 
			\underline{\textit{users}}\textsuperscript{\ref{db-table-users}} 
			Um prato pode ter apenas um utilizador que o criou (N:1).
		\item \textbf{\textit{establishments\_id}} - Identificador do estabelecimento na tabela 
			\underline{\textit{establishments}}\textsuperscript{\ref{db-table-establishments}} 
			Um prato pode ter apenas um estabelecimento onde existe (N:1).
	\end{itemize}

	\subsection{Tabela: \textit{Ratings}} \label{db-table-ratings}

	Tabela que contém os critérios de avaliação disponíveis na plataforma.

	\begin{itemize}
		\item \textbf{\textit{name}} - Nome do critério que será a identificação do que se está a avaliar
			na página respectiva.
		\item \textbf{\textit{description}} - Descrição do que o critério pretende avaliar para esclarecer
			o avaliador..
		\item \textbf{\textit{type}} - Tipo de critério:
			\begin{itemize}
				\item \textbf{0} - Critério do tipo numérico apenas.
				\item \textbf{1} - Critério do tipo numérico e textual.
				\item \textbf{2} - Critério do tipo textual apenas.
			\end{itemize}
		\item \textbf{\textit{enable}} - Permite desabilitar um critério na página de avaliações.
	\end{itemize}

\end{document}
